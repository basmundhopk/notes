\documentclass[11pt]{article}

% basic packages
\usepackage[margin=1in]{geometry}
\usepackage[pdftex]{graphicx}
\usepackage{amsmath,amssymb,amsthm}
\usepackage{custom}
\usepackage{lipsum}
\usepackage{enumitem}
% page formatting
\usepackage{fancyhdr}
\pagestyle{fancy}

\renewcommand{\sectionmark}[1]{\markright{\textsf{\arabic{section}. #1}}}
\renewcommand{\subsectionmark}[1]{}
\lhead{\textbf{\thepage} \ \ \nouppercase{\rightmark}}
\chead{}
\rhead{}
\lfoot{}
\cfoot{}
\rfoot{}
\setlength{\headheight}{14pt}

\linespread{1.03} % give a little extra room
\setlength{\parindent}{0.2in} % reduce paragraph indent a bit
\setcounter{secnumdepth}{2} % no numbered subsubsections
\setcounter{tocdepth}{2} % no subsubsections in ToC

\begin{document}

% make title page
\thispagestyle{empty}
\vspace*{\fill}
\vspace{0.1cm}
\begin{center}
{\fontsize{22}{22} \selectfont Practice Problems for}
\vskip 16pt
{\fontsize{36}{36} \selectfont \bf \sffamily grad div curl and all that}
\vskip 24pt
{\fontsize{18}{18} \selectfont \rmfamily Brett Hopkins} 
\vskip 6pt
{\fontsize{14}{14} \selectfont \ttfamily bhopkins@unr.edu} 
\vskip 24pt
\end{center}

\vspace*{\fill}

% make table of contents
\newpage
\microtoc
\newpage

% main content
\section{Chapter I}

\textbf{Problem I-1:}\\\\
\textbf{Problem I-2:}\\\\
\textbf{Problem I-3:}
\begin{enumerate}[label= \textbf{\Alph*.}]
	\item Write a formula for a vector function in two dimensions which is in the positive radial direction and whose magnitude is 1.\\
		
		\textbf{Solution:}
			\[
				\boxed{\frac{\mathbf{i} y + \mathbf{j} x}{\sqrt{x^2+y^2}}}
			\]
	\item Write a formula for a vector function in two dimensions whose direction makes an angle of
    $45^{\circ}$ with the x-axis and hwose magnitude at any point $(x,y)$ is $(x+y)^2$.\\

		\textbf{Solution:}
			\[
				\boxed{(\mathbf{i} + \mathbf{j}) \cdot \frac{(x + y)^2}{\sqrt{2}}}
			\]
	\item Write a formula for a vector function in two dimensions whose direction is tangential (in the sens of the example on page 5) and whose magnitude at any point (x,y) is equal to its distance from the origin. \\

		\textbf{Solution:}
		\[
			\boxed{-\mathbf{i}y + \mathbf{j}x}
		\]
	\item Write a formula for a vector function in three dimensions which is in the positive radial direction and whose magnitude is 1. \\
	
		\textbf{Solution:}
		\[
			\boxed{\frac{\mathbf{i}x+\mathbf{j}y+\mathbf{k}z}{\sqrt{x^2+y^2+z^2}}}
		\] 
\end{enumerate}

\textbf{Problem I-4:}
An object moves in the xy-plane in such a way that its position vector \textbf{r} is given as a function of time $t$ by 
\[
	\mathbf{r} = \mathbf{i}a\cos \omega t + \mathbf{j}b \sin \omega t
,\] 
where $a$, $b$, and $\omega$ are constants.

\begin{enumerate}[label= \textbf{\Alph*.}]
	\item How far is the object from the origin at any time $t$?\\

		\textbf{Solution:}
		\[
			\boxed{d = \sqrt{a^2\cos^2{\omega t} + b^2\sin^2{\omega t}}} 
		\] 

	\item Find the object's velocity and acceleration as functions of time.\\

		\textbf{Solution:}
		\[
			\boxed{\frac{dr}{dt} = -\mathbf{i} \omega a \sin{\omega t} + \mathbf{j} \omega b \cos{\omega t}}
		\]
		\[
			\boxed{\frac{d^2r}{dt^2} = -\mathbf{i}\omega^2a\cos{\omega t} - \mathbf{j}\omega^2b\sin{\omega t}}
		\]
	\item Show that the object moves on the elliptical path
		\[
		\biggl (\frac{x}{a} \biggr)^2 + \biggl(\frac{y}{b} \biggr)^2 = 1
		.\] 

		\textbf{Solution:}
		\[
			x(t) = a\cos{\omega t}
		.\] 
		\[
		y(t) = b\sin{\omega t}
		.\] 
		\[
			\biggl(\frac{a\cos \omega t}{a}\biggr)^2 + \biggl(\frac{b\sin \omega t}{b}\biggr)^2 = 1
		.\] 
		\[
		\cos^2 \omega t + \sin^2 \omega t = 1
		.\] 
		\[
			\boxed{1 = 1}
		\] 
\end{enumerate}

\textbf{Problem I-5:} A charge $+1$ is situated at the point $(1,0,0)$ and a charge $-1$ is situated at the point $(-1,0,0)$. Find the electric field of these two charges at an arbitrary point $(0,y,0)$ on the y-axis.\\

		\textbf{Solution:}
		\[
			E(r) = \frac{1}{4\pi \epsilon_0} \sum_{l=1}^{N}\frac{q q_0}{|r-r_l|^2}\mathbf{\hat{u}}_l
		.\]

		Contribution from $+1$ charge at $(1,0,0)$
		 \[
		 \mathbf{r} - \mathbf{r_1} = (0,y,0) - (1,0,0) = (-1,y,0).\] 
		\[
			|\mathbf{r} - \mathbf{r_1}| =  \sqrt{1 + y^2}
		.\]
		\[
			\mathbf{\hat{r}} = \frac{(-1,y,0)}{\sqrt{1 + y^2}}
		.\]

		Inside the summation for $+1$ charge at  $(1,0,0)$
		\[
		\frac{1}{(\sqrt{1 + y^2})^2} \cdot \frac{(-1,y,0)}{\sqrt{1+y^2}}
		.\] 
		\[
		= \frac{(-1,y,0)}{(1+y^2)^{\frac{3}{2}}}
		.\]

		Contribution from $-1$ charge at $(0,y,0)$
		\[
			- \frac{(1,y,0)}{(1+y^2)^{\frac{3}{2}}}
		.\] 

		Contribution from both charges
		\[
			E(r) = \frac{1}{4\pi\epsilon_0} \biggl(\frac{(-1,y,0)}{(1+y^2)^{\frac{3}{2}}} - \frac{(1,y,0)}{(1+y^2)^{\frac{3}{2}}}\biggr)
		.\] 
		\[
			E(r) = \frac{\mathbf{i}}{4\pi\epsilon_0} \cdot \frac{2}{(1+y^2)^{\frac{3}{2}}}
		.\] 
		\[
			\boxed{E(r)= \frac{\mathbf{i}}{2\pi\epsilon_0} \cdot \frac{1}{(1+y^2)^{\frac{3}{2}}}}
		\]

		\textbf{Problem I-6:} Instead of using arrows to represent vector functions, we sometimes use families of curves called \textit{field lines}. A curve $y=y(x)$ is a field line of the vector function  $\mathbf{F}(x, y)$ if at each point $(x_0 ,y_0)$ on the curve, $\mathbf{F} (x_0, y_0)$ is tangent to the curve.

\begin{enumerate}[label= \textbf{\Alph*.}]
	\item Show that the field lines $y = y(x)$ of a vector function
		\[
			\mathbf{F}(x,y) = \mathbf{i}F_x (x,y) + \mathbf{j} F_y (x,y)
		.\] 
	are solutions of the differential equation
	\[
	\frac{dy}{dx} = \frac{F_y (x,y)}{F_x (x,y)}
	.\]
	\textbf{Solution: }

Since the field line $y=y(x)$ has the vector function $\mathbf{F}(x,y)$ tangent to it at every point, the slope of the field line at any point must be equal to the slope of the vector at that point.
\[
	\text{slope of \textbf{F}} = \frac{\text{vertical component}}{\text{horizontal component}} = \frac{F_y (x,y)}{F_x(x,y)}
.\] 

Since the vector function is tangential to the field line at every point, their slopes must be equal.
\[
\frac{dy}{dx} = \frac{F_y (x,y)}{F_x (x,y)}
.\] 
Therefore, the field lines of the vector function are solutions of the differential equation. (This equation essentially states that the field line has the same slope at every point of the vector field, essentially the definition of a field line being tangent to a vector field.)

\item Determine the field lines of each of the funcitons of \textbf{Problem I-1}. Draw the field lines and compare with the arrow diagrams of \textbf{Problem I-1}.

	\begin{enumerate}[label= \textbf{\Alph*.}]
		\item $\mathbf{i}y + \mathbf{j}x$ \\

			\textbf{Solution:}
			\[
		F_x = y, \, F_y = x
		.\] 
		\[
		\frac{dy}{dx} = \frac{x}{y}
		.\] 
		\[
		\int y \, dy = \int x \, dx
		.\] 
		\[
			\frac{y^2}{2} = \frac{x^2}{2} + C
		.\] 
		\[
			y^2 = x^2 + C	
		.\] 
		\[
			\boxed{y^2-x^2 = C}
		\] 
	\item $(\mathbf{i} + \mathbf{j}) / \sqrt{2}$ \\

		\textbf{Solution:}
		\[
		F_x = \frac{1}{\sqrt{2}}, \, F_y = \frac{1}{\sqrt{2}}
	.\] 
		\[
		\frac{dy}{dx} = 1
		.\] 
		\[
		\int dy = \int dx
		.\] 
		\[
			\boxed{y = x + C}
		\] 
	\item $\mathbf{i}x - \mathbf{j}y$\\

		\textbf{Solution:}
		 \[
		F_x = x, \, F_y = -y
		.\] 
		\[
		\frac{dy}{dx} = -\frac{y}{x}
		.\] 
		\[
		\int \frac{1}{y} \, dy = -\int \frac{1}{x} \, dx
		.\] 
		\[
			\ln y = -\ln x + C
		.\] 
		\[
		\ln x + \ln y = C
		.\] 
\[
		\boxed{xy = C}
		\] 
	
	\item $\mathbf{i}y$\\

		\textbf{Solution:}
		\[
		F_x = y
		.\] 
		\[
		\frac{dy}{dx} = \frac{0}{y}
		.\] 
\[
	\boxed{y = C}
\] 
\item $\mathbf{j}x$\\

	\textbf{Solution:}
	 \[
	F_y = x
	.\] 
	\[
	\frac{dy}{dx} = \frac{x}{0} \, (undefined)
	.\] 
	\[
	dx = 0
	.\] 
	\[
		\boxed{x=C}
	\] 

\item $(\mathbf{i}y + \mathbf{j}x) / \sqrt{x^2 + y^2}, (x,y) \neq (0,0) $\\

	\textbf{Solution:}
	\[
	F_x = \frac{y}{\sqrt{x^2+y^2}}, \, F_y = \frac{x}{\sqrt{x^2 + y^2} }
	.\] 
	\[
	\frac{dy}{dx} = \frac{x}{y}
	.\] 
	\[
	\int y \, dy = \int x \, dx
	.\] 
	\[
	\frac{y^2}{2} = \frac{x^2}{2} + C
	.\] 
	\[
		\boxed{y^2 - x^2 = C}
	\] 
\item $\mathbf{i}y + \mathbf{j}xy$\\

	\textbf{Solution:}
	\[
	F_x = y, \, F_y = xy
	.\] 
	\[
	\frac{dy}{dx} = \frac{xy}{y}
	.\] 
	\[
	\frac{dy}{dx} = x
	.\] 
	\[
	\int dy = \int x \, dx
	.\] 
	\[
		\boxed{y = \frac{x^2}{2} + C}
	\] 
\item $\mathbf{i} + \mathbf{j}y$\\

	\textbf{Soltuion:}
	 \[
	F_x = 1, \, F_y = y
	.\] 
	\[
	\frac{dy}{dx} = y
	.\] 
	\[
	\int \frac{1}{y} \, dy = \int dx
	.\] 
	\[
	\ln y = x + C
	.\] 
	\[
	\boxed{y = e^x + C}
	\]

	\end{enumerate}
\end{enumerate}

\newpage
\section{Chapter II}
\textbf{Problem II-1:}\\
Find a unit vector $\mathbf{\hat{n}}$ normal to each of the following surfaces. \\

In general,
  \[
  \mathbf{\hat{n}}(x,y,z) = \frac{-\mathbf{i} \frac{\partial z}{\partial x} - \mathbf{j} \frac{\partial z}{\partial y} + \mathbf{k}}{\sqrt{1 + \left(\frac{\partial z}{\partial x}\right)^2 + \left(\frac{\partial z}{\partial y}\right)^2}}
  .\]

\begin{enumerate}[label= \textbf{\Alph*.}]
  \item $z = 2-x-y$ \\

    \textbf{Solution:}
    \[
      \frac{\partial z}{\partial x} = -1
    .\]
    \[
      \frac{\partial z}{\partial y} = -1 
    .\]
    \[
      \boxed{\mathbf{\hat{n}} = \frac{\mathbf{i} + \mathbf{j} + \mathbf{k}}{\sqrt{3}}}
    \]

  \item $z = (x^2 + y^2)^{\frac{1}{2}}$\\

    \textbf{Soltuion:}
    \[
      \frac{\partial z}{\partial x} = 2x \frac{1}{\sqrt{x^2 + y^2}} = \frac{x}{z}
    .\]
    \[
      \frac{\partial z}{\partial y} = 2y \frac{1}{\sqrt{x^2+y^2}} = \frac{y}{z}
    .\]
    \[
      \mathbf{\hat{n}} = \frac{-\mathbf{i} \frac{x}{z} - \mathbf{j} \frac{y}{z} + \mathbf{k}}{\sqrt{1 + \left(\frac{x}{z}\right)^2 + \left(\frac{y}{z}\right)^2}}
    .\]
    \[
      \boxed{\mathbf{\hat{n}} = \frac{-\mathbf{i} x - \mathbf{j} y + \mathbf{k} z}{\sqrt{2}}}
    \]
  \item $z = \sqrt{1-x^2}$ \\

    \textbf{Solution:}
  \[
    \frac{\partial z}{\partial x} = \frac{1}{2} \cdot -2x \cdot \frac{1}{\sqrt{1-x^2} } = \frac{-x}{\sqrt{1-x^2} }
  .\] 
  \[
    \frac{\partial z}{\partial y} = 0
  .\]
  \[
    \frac{-\mathbf{i} \frac{-x}{\sqrt{1-x^2} } + \mathbf{k}}{\sqrt{1+(\frac{-x}{\sqrt{1-x^2} })^2} }
  .\]
  \[
    \frac{\mathbf{i} \frac{x}{z }+ \mathbf{k}}{\sqrt{ 1+\frac{x^2}{1-x^2}}}
  .\] 
  \[
    \frac{\mathbf{i} \frac{x}{z} + \mathbf{k}}{\sqrt{ \frac{1-x^2+x^2}{1-x^2}}}
  .\] 
  \[ 
    \frac{\mathbf{i} \frac{x}{z} + \mathbf{k}}{\sqrt{ \frac{1}{z^2}}}
  .\]
  \[
    \frac{\mathbf{i} \frac{x}{z} + \mathbf{k}}{\frac{1}{z}}
  .\] 
  \[
    \boxed{\mathbf{i} x + \mathbf{k} z} 
  \] 
  \item $z = x^2 + y^2$\\

    \textbf{Solution:}
  \[
    \frac{\partial z}{\partial x} = 2x 
  .\] 
  \[
    \frac{\partial z}{\partial y} = 2y
  .\] 
  \[ 
    \frac{-\mathbf{i} 2x - \mathbf{j} 2y + \mathbf{k}}{\sqrt{1 + (2x)^2 + (2y)^2}}
  .\]
  \[
    \frac{-\mathbf{i}2x-\mathbf{j}2y+\mathbf{k}}{\sqrt{1+4x^2+4y^2} }
  \] 
  \[
    \boxed{\frac{-\mathbf{i}2x-\mathbf{j}2y+\mathbf{k}}{\sqrt{1+4z}}}
  \]
\item $z = \biggl(1-\frac{x^2}{a^2} - \frac{y^2}{a^2}\biggr)^{\frac{1}{2}}$ \\

  \textbf{Solution:}
  \[
    \frac{\partial z}{\partial x} = \frac{1}{2} - \frac{2x}{a^2} \cdot \frac{1}{\sqrt{1-\frac{x^2}{a^2} - \frac{y^2}{a^2}} } = \frac{x}{a^2z}
  \] 
  \[
    \frac{\partial z}{\partial y} = \frac{y}{a^2z}
  \] 
  \[
    \mathbf{\hat{n}} = \frac{-\mathbf{i}\frac{x}{a^2z} - \mathbf{j}\frac{y}{a^2z} + \mathbf{k}}{\sqrt{1+(\frac{x}{a^2z})^2 + (\frac{y}{a^2z})^2} }
  \] 
  \[
    \mathbf{\hat{n}} = \frac{-\mathbf{i}x-\mathbf{j}y+\mathbf{k}a^2z}{\sqrt{1+\frac{x^2}{a^4 z^2} + \frac{y^2}{a^4z^2}}} = \frac{-\mathbf{i}x-\mathbf{j}y + \mathbf{k}a^2z}{\sqrt{1 + \frac{x^2+y^2}{a^4z^2}}}
  \] 
  \[

  \] 
\end{enumerate}


\end{document}

